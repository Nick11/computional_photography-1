\documentclass{paper}


\usepackage{epsfig}
\usepackage{graphicx}
\usepackage{amsmath}
\usepackage{amssymb}
\usepackage{amsthm}

\usepackage{color}
\usepackage{subcaption}
\usepackage{caption}




\usepackage{mathrsfs}
\usepackage{algpseudocode}
\usepackage{algorithm}

\usepackage{float}
\usepackage{mathtools}
\usepackage{mdwlist}
\usepackage{gensymb}
\usepackage{array}
\usepackage{multirow}
\usepackage[hmargin=3cm]{geometry}
\usepackage{boxedminipage}
\usepackage{enumerate}


\setlength{\parindent}{0pt}
\setlength{\parskip}{18pt}


\usepackage[latin1]{inputenc} 
\usepackage[T1]{fontenc} 

\usepackage{listings} 
\lstset{% 
   language=Matlab, 
   basicstyle=\small\ttfamily, 
} 






\renewcommand{\algorithmicforall}{\textbf{Foreach}}
\newcommand{\init}{\textbf{INIT }}
\newcommand{\pluseq}{\mathrel{+}=}
\newcommand{\asteq}{\mathrel{*}=}
\newcommand{\myto}{\textbf{TO }}
\newcommand*\colvec[3][]{
    \begin{pmatrix}\ifx\relax#1\relax\else#1\\\fi#2\\#3\end{pmatrix}
}
\newcommand{\myparagraph}[1]{\paragraph{#1}\mbox{}\\}
\DeclarePairedDelimiter\ceil{\lceil}{\rceil}
\DeclarePairedDelimiter\floor{\lfloor}{\rfloor}



\title{Computational Photography Assignment 3}



\author{Single Michael\\08-917-445}
% //////////////////////////////////////////////////


\begin{document}



\maketitle


\section*{Task 1}

\paragraph{Def(Linear)} Given two measurable function $f$ and $g$ and $\alpha$, $\beta \in \mathbb{C}$. An operator $\mathcal{F}$ is called $\emph{linear}$ if the following identity holds true: \\
\begin{equation}
    \mathcal{F} \left \{\alpha f + \beta g \right\}(x) = \alpha \mathcal{F} \left \{f\right\}(x) + \beta \mathcal{F} \left \{g \right\}(x)
\label{eq:lin_op}
\end{equation}

\paragraph{Def(Shift Invariant)} Given a measurable function $f$. An operator $\mathcal{F}$ is called $\emph{shift invariant}$ if the following identity holds true: \\
\begin{equation}
    \mathcal{F} \left \{S_\delta f \right\} (x) = S_\delta(\mathcal{F}\left \{f \right\})(x)
\label{eq:shift_inv_op}
\end{equation}

where the shift operator $S_{\delta}$ is defined as the following: 

\begin{equation}
    S_{\delta}(f)(x) = f(x+\delta)
\label{eq:shift_op}
\end{equation}

For any given function $f$ and small number $\delta \in \mathbb{R}$. \\

Note that notational convention the operation notion for any operator $F$, we write $F(f)(x)$ instead of $F(f(x))$. In words we apply the operator F (i.e. the functional F) to the function $f$. Thus is would even be possible to omit the argument $x$. However to make the reasoning clear, I will keep it. \\

Therefore, directly applying the definition from equation $\ref{eq:shift_op}$ to equation $\ref{eq:shift_inv_op}$ we can also define the property $\emph{shift invariant}$ as the follwoing:

\begin{equation}
    \mathcal{F} \left \{S_\delta (f(x)) \right\} := \mathcal{F} \left\{ f(x+\delta) \right\} = S_\delta(\mathcal{F}\left \{f \right\}(x)
\label{eq:shift_inv_op_easy}
\end{equation}

Relying on the definitions of the equations before, equation $\ref{eq:shift_inv_op_easy}$ and equation $\ref{eq:lin_op}$ let I solved this task,

\begin{enumerate}[(a)] 

% a)
\item $\mathcal{F} \left \{ f \right\} (x) = e^{f(x)}$ is $\emph{not linear}$ and is $\emph{shift invariant}$. 

\paragraph{Show:} $\mathcal{F}$ is not \emph{linear} \\
\begin{proof}
\begin{align*}
    \mathcal{F} \left \{\alpha f + \beta g \right\}(x) 
    &= e^{\alpha f(x) + \beta g(x)} \\
    &= e^{\alpha f(x)} e^{\beta g(x)}\\
    &\neq \alpha e^{f(x)} + \beta e^{g(x)}\\
    &= \alpha \mathcal{F} \left \{f\right\}(x) + \beta \mathcal{F} \left \{g \right\}(x)
\end{align*}
\end{proof}

\paragraph{Show} $\mathcal{F}$ is \emph{shift invariant} \\
\begin{proof}
\begin{align*}
    \mathcal{F} \left \{S_\delta (f(x)) \right\} 
    &= \mathcal{F} \left\{ f(x+\delta) \right\} \\
    &= e^{f(x + \delta)} \\
    &= S_{\delta}(e^{f(x)}) \\
    &= S_\delta(\mathcal{F}\left \{f \right\}(x)
\end{align*}
\end{proof}


% b)
\item $\mathcal{F} \left \{ f \right\} (x) = f(x)f(x-1)$ is $\emph{not linear}$ and is $\emph{shift invariant}$.

\paragraph{Show} $\mathcal{F}$ is not \emph{linear} \\
\begin{proof}
\begin{align*}
    \mathcal{F} \left \{\alpha f + \beta g \right\}(x)
    &= (\alpha f(x) + \beta g(x))(\alpha f(x-1) + \beta g(x-1)) \\
    &= \alpha^2 f(x)f(x-1) + \beta^2 g(x)g(x-1) + \alpha \beta (f(x)g(x-1)+f(x-1)g(x)) \\
    &\neq \alpha f(x)f(x-1) + \beta g(x)g(x-1) \\
    &= \alpha \mathcal{F} \left \{f\right\}(x) + \beta \mathcal{F} \left \{g \right\}(x)
\end{align*}
\end{proof}

\paragraph{Show} $\mathcal{F}$ is \emph{shift invariant} \\
\begin{proof}
\begin{align*}
    \mathcal{F} \left \{S_\delta (f(x)) \right\} 
    &= \mathcal{F} \left\{ f(x+\delta) \right\} \\
    &= f(x+\delta)f(x-1+\delta) \\
    &= S_{\delta}(f(x)f(x-1)) \\
    &= S_\delta(\mathcal{F}\left \{f \right\}(x)
\end{align*}
\end{proof}

% c)
\item $\mathcal{F} \left \{ f \right\} (x) = \sum_{k=x-1}^{x+2}f(k)$ is $\emph{linear}$ and is $\emph{shift invariant}$.

\paragraph{Show} $\mathcal{F}$ is \emph{linear} \\
\begin{proof}
\begin{align*}
    \mathcal{F} \left \{\alpha f + \beta g \right\}(x)
    &= \sum_{k=x-1}^{x+2}(\alpha f(k) + \beta g(k)) \\
    &= \alpha \sum_{k=x-1}^{x+2} f(k) + \beta \sum_{k=x-1}^{x+2} g(k) \\
    &= \alpha \mathcal{F} \left \{f\right\}(x) + \beta \mathcal{F} \left \{g \right\}(x)
\end{align*}
\end{proof}

\paragraph{Show} $\mathcal{F}$ is \emph{shift invariant} \\
\begin{proof}
\begin{align*}
    \mathcal{F} \left \{S_\delta (f(x)) \right\} 
    &= \mathcal{F} \left\{ f(x+\delta) \right\} \\
    &= \sum_{k=x-1+\delta}^{x+2+\delta} f(k) \\
    &= S_{\delta}(\sum_{k=x-1}^{x+2} f(k)) \\
    &= S_\delta(\mathcal{F}\left \{f \right\}(x)
\end{align*}
\end{proof}

% d)
\item $\mathcal{F} \left \{ f \right\} (x) = f(2x)$ is $\emph{linear}$ and is $\emph{not shift invariant}$.

\paragraph{Show} $\mathcal{F}$ is \emph{linear} \\
\begin{proof}
\begin{align*}
    \mathcal{F} \left \{\alpha f + \beta g \right\}(x)
    &= \alpha f(2x) + \beta g(2x) \\
    &= \alpha \mathcal{F} \left \{f\right\}(x) + \beta \mathcal{F} \left \{g \right\}(x)
\end{align*}
\end{proof}

\paragraph{Show} $\mathcal{F}$ is not \emph{shift invariant} \\
\begin{proof}
\begin{align*}
    \mathcal{F} \left \{S_\delta (f(x)) \right\} 
    &= \mathcal{F} \left\{ f(x+\delta) \right\} \\
    &= f(2(x+\delta)) \\
    &= f(2x + 2\delta) \\
    &\neq f(2x + \delta) \\
    &= S_\delta (f(2x)) \\
    &= S_\delta(\mathcal{F}\left \{f \right\}(x)
\end{align*}
\end{proof}


\item $\mathcal{F} \left \{ f \right\} (x) = s$ is $\emph{not linear}$ and is $\emph{shift invariant}$.

\paragraph{Show} $\mathcal{F}$ is \emph{linear} \\
\begin{proof}
\begin{align*}
    a = a
\end{align*}
\end{proof}

\paragraph{Show} $\mathcal{F}$ is \emph{shift invariant} \\
\begin{proof}
\begin{align*}
    a = a
\end{align*}
\end{proof}


\item $\mathcal{F} \left \{ f \right\} (x) = s$ is $\emph{not linear}$ and is $\emph{shift invariant}$.

\paragraph{Show} $\mathcal{F}$ is \emph{linear} \\
\begin{proof}
\begin{align*}
    a = a
\end{align*}
\end{proof}

\paragraph{Show} $\mathcal{F}$ is \emph{shift invariant} \\
\begin{proof}
\begin{align*}
    a = a
\end{align*}
\end{proof}

\item $\mathcal{F} \left \{ f \right\} (x) = s$ is $\emph{not linear}$ and is $\emph{shift invariant}$.

\paragraph{Show} $\mathcal{F}$ is \emph{linear} \\
\begin{proof}
\begin{align*}
    a = a
\end{align*}
\end{proof}

\paragraph{Show} $\mathcal{F}$ is \emph{shift invariant} \\
\begin{proof}
\begin{align*}
    a = a
\end{align*}
\end{proof}

\item $\mathcal{F} \left \{ f \right\} (x) = s$ is $\emph{not linear}$ and is $\emph{shift invariant}$.

\paragraph{Show} $\mathcal{F}$ is \emph{linear} \\
\begin{proof}
\begin{align*}
    a = a
\end{align*}
\end{proof}

\paragraph{Show} $\mathcal{F}$ is \emph{shift invariant} \\
\begin{proof}
\begin{align*}
    a = a
\end{align*}
\end{proof}


\end{enumerate}

    
\section*{Task 2}

Given an $m \times n$ monochromatic (i.e. there is only one color-channel) Image $I$. 
Give an algorithm how to apply box-filtering on this image. Furthermore analyse the asymptotic complexity of this algorithm.

\begin{algorithm}[H]
\caption{Moving Average box filter}
\begin{table}[H]
  \begin{tabular}{@{}lll@{}}
    \textbf{Input:} & Grayscale \emph{Image} I with resolution $m \times n$  \\
    \textbf{Output:} & Box filtered Image \emph{Image} $\hat{I}$   \\
  \end{tabular} 
\end{table}
\textbf{Procedures:} $getDimensions(Image)$, $zeros(height, width)$  \\
\setlength{\fboxrule}{0pt} 
\begin{boxedminipage}{1.0\textwidth}
  \begin{algorithmic}[1]
      \State $ [h,w] = getDimensions(I)$
      \State $ \hat{I} = zeros(h,w)$
      \State $ r = \ceil{\frac{w-1}{2}}$
      \ForAll{$Pixel \thinspace p \in Image \thinspace I$}
        \State $ contribution = 0$
        \ForAll{$Pixel \thinspace p_n \in r-Neighborhood \thinspace \ \mathcal{N}_r(p)$}
            \State $ contribution = contribution + I(p+p_n)$
        \EndFor
        \State $ \hat{I}(p) = \frac{contribution}{m \cdot n}$
      \EndFor
  \end{algorithmic}
  \end{boxedminipage}
  \vskip1.5pt
\label{alg:boxfilter}
\end{algorithm}

\paragraph{Remarks:}

\begin{itemize}
    \item By pixels in the Algorithm $\ref{alg:boxfilter}$ we are referring to the coordinates of the pixel in the image. Therefore $p$ corresponds to the $x$ and $y$ coordinates of pixel $p$ in the Image $I$.
    \item $I(p)$ denotes accessing the pixel-(color)-values in the images at the position of the pixel $p$ in the image $I$.
    \item $\mathcal{N}_r(p)$ denotes the neighborhood with radius $r$ around a given pixel $p$. In the context of pixel-coordinates, think of it as a box-grid, centred at the pixel coordinates of $p$. This grid has a radius of r. This means there are $r$ neighbors (pixel-coordinates in the grid) below, on top, on the left and on the right of $p$.
    \item Our algorithm can easily be extended for color Images by simply applying the same algorithm to each color-channel separately.
    \item The assumption of being provided by a m by n can easily be extended for the case when $n \neq m$. This only will affect the computation of the radius $r$ in algorithm $\ref{alg:boxfilter}$. \\ Computing $\ceil{0.5 \cdot \left( \ceil{\frac{m-1}{2}} + \ceil{\frac{n-1}{2}}\right)}$ would be a valid option in order to compute $r$. 
    \item If $w$ (i.e. n) is odd, then $\ceil{\frac{w-1}{2}}$ is equal to $\frac{w-1}{2}$. 
    \item The procedure $getDimensions$ returns the width-and height resolution of a provided Image.
    \item The procedure $zeros$ creates a new image with the provided resolutions.
\end{itemize}


\paragraph{Aysmptotic Complexity}


\section*{Task 3}
\section*{Task 4}
\section*{Task 5}




\section*{Task 6}



\end{document}